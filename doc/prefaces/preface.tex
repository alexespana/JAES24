\noindent{\textbf{Resumen --}}
En el contexto del desarrollo de \textit{software} moderno, la Integración Continua (\textit{CI}) es
una práctica ampliamente adoptada que busca automatizar el proceso de integración de cambios de
código en un proyecto. A pesar de ofrecer numerosas ventajas, implementarla conlleva una serie de
costos significativos que deben ser abordados para garantizar la eficiencia a largo plazo. La fase
de Integración Continua puede resultar costosa tanto en términos de recursos computacionales como
económicos, llevando a grandes empresas como Google y Mozilla a invertir millones de dólares en
sus sistemas de \textit{CI} \cite{1}. Han surgido numerosos enfoques para reducir el costo asociado
a la carga computacional evitando ejecutar construcciones que se espera que sean exitosas
\cite{2}. Sin embargo, estos enfoques no son precisos, llegando a  hacer predicciones erróneas que
omiten ejecutar construcciones que realmente fallan. Además de los costos asociados con la carga
computacional y económica de la \textit{CI}, otro problema al que se enfrentan los equipos de
desarrollo de \textit{software} es el tiempo que deben esperar para obtener \textit{feedback} del
resultado del proceso de \textit{CI} \cite{3}. Este tiempo de espera en ocasiones puede ser
significativo y puede afectar negativamente a la productividad y eficiencia del equipo, así como
a la capacidad de respuesta ante problemas y ajustes rápidos en el desarrollo. Así, en este
trabajo nuestro objetivo es reducir el costo computacional en \textit{CI}, al mismo tiempo que
maximizamos la observación de construcciones fallidas. Para ello, se ha realizado un estudio
sobre las técnicas existentes \cite{2,4,5,6,7,8}, y se ha propuesto una implementación,
\textit{JAES24}, que busca contribuir a las mismas. Este nuevo enfoque amplia el estado del arte
de técnicas existentes que hacen uso de \textit{Machine Learning} para la predicción de construcciones
fallidas, mejorando sus resultados y ofreciendo un punto diferenciador, una interfaz gráfica. Dicha
interfaz permite interactuar de forma sencilla con el sistema, abstrayendo la complejidad de los
algoritmos de predicción y ofreciendo una forma intuitiva y sencilla de realizar predicciones basadas
en un repositorio concreto. Posteriormente, se han realizado una serie de experimentos para verificar
y validar la efectividad de \textit{JAES24} en comparación con otras técnicas existentes. Finalmente,
se desarrollarán unas conclusiones sobre lo resultados obtenidos y se propondrán posibles líneas de
trabajo futuro.

\vspace{0.5cm}
\noindent{\textbf{Palabras clave}: Integración Continua \and Predicción de Builds 
                                    \and Aprendizaje Automático \and Ahorro de Costos
                                    \and Características de Builds} 

\vspace{1cm}

\noindent{\textbf{Abstract --}}
In the context of modern software development, Continuous Integration (CI) is a widely adopted
practice that aims to automate the process of integrating code changes in a project. Despite
offering numerous advantages, implementing CI involves significant costs that need to be addressed
to ensure long-term efficiency. The Continuous Integration phase can be costly in terms of
computational and economic resources, leading large companies like Google and Mozilla to invest
millions of dollars in their CI systems \cite{1}. Several approaches have emerged to reduce the
cost associated with computational load by avoiding running builds that are expected to be
successful \cite{2}. However, these approaches are not accurate, often making erroneous
predictions that skip running builds that actually fail. In addition to the costs associated with
computational and economic load of CI, another problem faced by software development teams is the
time they have to wait to get feedback on the CI process outcome \cite{3}. This waiting time can
sometimes be significant and can negatively impact team productivity and efficiency, as well as
the ability to respond to issues and make quick adjustments in development. Therefore, the
objective of this work is to reduce the computational cost in CI while maximizing the observation
of failed builds. To achieve this, a study has been conducted on existing techniques
\cite{2,4,5,6,7,8}, and an implementation, \textit{JAES24}, has been proposed to contribute to them.
This new approach extends the state of the art of existing techniques that use
\textit{Machine Learning} for predicting build failures, improving their results and offering a
distinguishing feature, a graphical interface. This interface allows for easy interaction with the
system, abstracting the complexity of the prediction algorithms and providing an intuitive and simple
way to make predictions based on a specific repository. Subsequently, a series of experiments have
been conducted to verify and validate the effectiveness of \textit{JAES24} in comparison with other
existing techniques. Finally, conclusions will be drawn from the obtained results and possible
future work lines will be proposed.

\vspace{0.5cm}
\noindent{\textbf{Keywords}: Continuous Integration \and Build Prediction 
                                \and Machine Learning \and Cost Saving
                                \and Build Features}

\vfill
