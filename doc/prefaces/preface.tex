\noindent{\textbf{Resumen --}}
En el contexto del desarrollo de software moderno, la Integración Continua (\textit{CI}) es una 
práctica ampliamente adoptada que busca automatizar el proceso de integración de cambios de código
en un proyecto. A pesar de ofrecer numerosas ventajas, implementarla conlleva una serie de costos
significativos que deben ser abordados para la garantizar la eficiencia a largo plazo. La fase de
Integración Continua puede resultar costosa tanto en términos de recursos computacionales como
económicos, llevando a grandes empresas como Google y Mozilla a invertir millones de dólares en
sus sistemas de \textit{CI} \cite{1}. Han surgido numerosos enfoques para reducir el costo asociado
a la carga computacional evitando ejecutar construcciones que se espera que sean exitosas
\cite{2}. Sin embargo, estos enfoques no son exactos, llegando a  hacer predicciones erróneas que
omiten ejecutar construcciones que realmente fallan. Además de los costos asociados con la carga
computacional y económica de la \textit{CI}, otro problema al que se enfrentan los equipos de
desarrollo de software es el tiempo que deben esperar para obtener \textit{feedback} del
resultado del proceso de \textit{CI} \cite{3}. Este tiempo de espera en ocasiones puede ser
significativo y puede afectar negativamente a la productividad y eficiencia del equipo, así como
a la capacidad de respuesta ante problemas y ajustes rápidos en el desarrollo. Así, en este
trabajo nuestro objetivo es reducir el costo computacional en CI, al mismo tiempo que maximizamos
la observación de construcciones fallidas. Para ello, se ha realizado un estudio empírico sobre
técnicas existentes \cite{2,4,5,6,7,8}, y se ha propuesto una implementación, \textit{JAES24}, que
busca contribuir a las mismas. Posteriormente, se han realizado una serie de experimentos para
verificar y validar la efectividad de \textit{JAES24} en comparación con otras técnicas existentes.
Finalmente, se desarrollarán unas conclusiones sobre los resultados obtenidos y se propondrán
posibles líneas de trabajo futuro.

\vspace{0.5cm}
\noindent{\textbf{Palabras clave}: Integración Continua \and Predicción de Builds 
                                    \and Costo de Mantenimiento \and Aprendizaje Automático 
                                    \and Fallas de Builds \and Predicción de Fallas 
                                    \and Ahorro de Costos \and Predicción de Resultados de Builds
                                    \and Mantenimiento de Software \and Características de Builds} 

\vspace{2cm}

\noindent{\textbf{Abstract --}}
In the context of modern software development, Continuous Integration (CI) is a widely adopted
practice that aims to automate the process of integrating code changes in a project. Despite
offering numerous advantages, implementing CI involves significant costs that need to be addressed
to ensure long-term efficiency. The Continuous Integration phase can be costly in terms of
computational and economic resources, leading large companies like Google and Mozilla to invest
millions of dollars in their CI systems \cite{1}. Several approaches have emerged to reduce the
cost associated with computational load by avoiding running builds that are expected to be
successful \cite{2}. However, these approaches are not accurate, often making erroneous
predictions that skip running builds that actually fail. In addition to the costs associated with
computational and economic load of CI, another problem faced by software development teams is the
time they have to wait to get feedback on the CI process outcome \cite{3}. This waiting time can
sometimes be significant and can negatively impact team productivity and efficiency, as well as
the ability to respond to issues and make quick adjustments in development. Therefore, the
objective of this work is to reduce the computational cost in CI while maximizing the observation
of failed builds. To achieve this, an empirical study on existing techniques has been conducted
\cite{2,4,5,6,7,8}, and an implementation, \textit{JAES24}, has been proposed to contribute to
these techniques. Subsequently, a series of experiments have been conducted to verify and validate
the effectiveness of \textit{JAES24} compared to other existing techniques. Finally, conclusions
will be drawn on the obtained results and possible future lines of work will be proposed.

\vspace{0.5cm}
\noindent{\textbf{Keywords}: Continuous Integration \and Build Prediction 
                                \and Maintenance Cost \and Machine Learning
                                \and Build Failures \and Failure Prediction
                                \and Cost Saving \and Build Outcome Prediction
                                \and Software Maintenance \and Build Features}

\vfill
