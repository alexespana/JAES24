\newpage
\begin{thebibliography}{8}

\bibitem{1}
Xianhao Jin and Francisco Servant. 2023. HybridCISave: A Combined Build and Test Selection
Approach in Continuous Integration. ACM Trans. Softw. Eng. Methodol. 32, 4, Article 93 (July
2023), 39 pages. https://doi.org/10.1145/3576038

\bibitem{2}
Xianhao Jin and Francisco Servant. 2020. A cost-efficient approach to building in continuous
integration. In Proceedings of the ACM/IEEE 42nd International Conference on Software Engineering
(ICSE '20). Association for Computing Machinery, New York, NY, USA, 13–25.
https://doi.org/10.1145/3377811.3380437

\bibitem{3}
Xianhao Jin and Francisco Servant. 2021. CIBench: a dataset and collection of techniques for
build and test selection and prioritization in continuous integration. In Proceedings of the
43rd International Conference on Software Engineering: Companion Proceedings (ICSE '21). IEEE
Press, 166–167. https://doi.org/10.1109/ICSE-Companion52605.2021.00070

\bibitem{4}
Xianhao Jin and Francisco Servant. 2022. Which builds are really safe to skip? Maximizing failure
observation for build selection in continuous integration. J. Syst. Softw. 188, C (Jun 2022).
https://doi.org/10.1016/j.jss.2022.111292

\bibitem{5}
Islem Saidani, Ali Ouni, Moataz Chouchen, and Mohamed Wiem Mkaouer. 2020. Predicting continuous
integration build failures using evolutionary search. Inf. Softw. Technol. 128, C (Dec 2020).
https://doi.org/10.1016/j.infsof.2020.106392

\bibitem{6}
Xianhao Jin and Francisco Servant. 2023. HybridCISave: A Combined Build and Test Selection
Approach in Continuous Integration. ACM Trans. Softw. Eng. Methodol. 32, 4, Article 93 (July
2023), 39 pages. https://doi.org/10.1145/3576038

\bibitem{7}
Bihuan Chen, Linlin Chen, Chen Zhang, and Xin Peng. 2021. BuildFast: history-aware build outcome
prediction for fast feedback and reduced cost in continuous integration. In Proceedings of the
35th IEEE/ACM International Conference on Automated Software Engineering (ASE '20). Association
for Computing Machinery, New York, NY, USA, 42–53. https://doi.org/10.1145/3324884.3416616

\bibitem{8}
Foyzul Hassan and Xiaoyin Wang. 2017. Change-aware build prediction model for stall avoidance in
continuous integration. In Proceedings of the 11th ACM/IEEE International Symposium on Empirical
Software Engineering and Measurement (ESEM '17). IEEE Press, 157–162.
https://doi.org/10.1109/ESEM.2017.23

\bibitem{9}
O. Elazhary, C. Werner, Z. S. Li, D. Lowlind, N. A. Ernst and M. -A. Storey, "Uncovering the
Benefits and Challenges of Continuous Integration Practices," in IEEE Transactions on Software
Engineering, vol. 48, no. 7, pp. 2570-2583, 1 July 2022, doi: 10.1109/TSE.2021.3064953.

\bibitem{10}
Rostami Mazrae P., Mens T., Golzadeh M., Decan A. On the usage, co-usage and migration of CI/CD
tools: A qualitative analysis (2023) Empirical Software Engineering, 28 (2), art. no. 52, Cited
15 times. DOI: 10.1007/s10664-022-10285-5

\bibitem{11}
M. Hilton, T. Tunnell, K. Huang, D. Marinov and D. Dig, "Usage, costs, and benefits of continuous
integration in open-source projects," 2016 31st IEEE/ACM International Conference on Automated
Software Engineering (ASE), Singapore, 2016, pp. 426-437.

\bibitem{12}
Michael Hilton, Timothy Tunnell, Kai Huang, Darko Marinov, and Danny Dig. 2016. Usage, costs, and
benefits of continuous integration in open-source projects. In Proceedings of the 31st IEEE/ACM
International Conference on Automated Software Engineering (ASE '16). Association for Computing
Machinery, New York, NY, USA, 426–437. https://doi.org/10.1145/2970276.2970358

\bibitem{13}
Kim Herzig, Michaela Greiler, Jacek Czerwonka, and Brendan Murphy. 2015. The art of testing less
without sacrificing quality. In Proceedings of the 37th International Conference on Software
Engineering - Volume 1 (ICSE '15). IEEE Press, 483–493.

\bibitem{14}
Hilton, Michael; Tunnell, Timothy; Huang, Kai ; Marinov, Darko ; Dig, Danny. Usage, costs, and
benefits of continuous integration in open-source projects (2016) ASE 2016 - Proceedings of the
31st IEEE/ACM International Conference on Automated Software Engineering, pp. 426 - 437
DOI: 10.1145/2970276.2970358

\bibitem{15}
M. Fowler and M. Foemmel, “Continuous integration,” https://tinyurl.com/ycbl2uhj, 2006, 
[Online; accessed 2-Aug-2023].

\bibitem{16}
Thomas Rausch, Waldemar Hummer, Philipp Leitner, and Stefan Schulte. 2017. An empirical analysis
of build failures in the continuous integration workflows of Java-based open-source software. In
Proceedings of the 14th International Conference on Mining Software Repositories (MSR '17). IEEE
Press, 345–355. https://doi.org/10.1109/MSR.2017.54

\bibitem{17}
Islem Saidani, Ali Ouni, Moataz Chouchen, and Mohamed Wiem Mkaouer. 2021. BF-detector: an
automated tool for CI build failure detection. In Proceedings of the 29th ACM Joint Meeting on
European Software Engineering Conference and Symposium on the Foundations of Software Engineering
(ESEC/FSE 2021). Association for Computing Machinery, New York, NY, USA, 1530–1534.
https://doi.org/10.1145/3468264.3473115

\bibitem{18}
Islem Saidani, Ali Ouni, and Mohamed Wiem Mkaouer. 2022. Improving the prediction of continuous
integration build failures using deep learning. Automated Software Engg. 29, 1 (May 2022).
https://doi.org/10.1007/s10515-021-00319-5

\bibitem{19}
Md Rakibul Islam and Minhaz F. Zibran. 2017. Insights into continuous integration build failures.
In Proceedings of the 14th International Conference on Mining Software Repositories (MSR '17).
IEEE Press, 467–470. https://doi.org/10.1109/MSR.2017.30

\bibitem{20}
Hong, Yang; Tantithamthavorn, Chakkrit; Pasuksmit, Jirat; Thongtanunam, Patanamon; Friedman, Arik;
Zhao, Xing; Krasikov, Anton; Practitioners’ Challenges and Perceptions of CI Build Failure
Predictions at Atlassian (2024), pp. 370 - 381. DOI: 10.1145/3663529.3663856

\bibitem{21}
Eriks Klotins, Tony Gorschek, Katarina Sundelin, and Erik Falk. 2022. Towards cost-benefit
evaluation for continuous software engineering activities. Empirical Softw. Engg. 27, 6 (Nov
2022). https://doi.org/10.1007/s10664-022-10191-w

\bibitem{22}
B. Liu, et al.,"The Why, When, What, and How About Predictive Continuous Integration: A
Simulation-Based Investigation" in IEEE Transactions on Software Engineering, vol. 49, no. 12, pp.
5223-5249, 2023. doi: 10.1109/TSE.2023.3330510

\end{thebibliography}