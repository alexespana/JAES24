\section{Introducción}
La Integración Continua (\textit{Continuous Integration, CI}) es una práctica de desarrollo de
\textit{software} que busca automatizar el proceso de fusión de cambios de código en un proyecto,
donde cada integración es verificada mediante la ejecución automática de pruebas. Este proceso
busca la detección temprana de errores y mejorar la calidad del software, permitiendo una
integración más frecuente y rápida del trabajo de todos los desarrolladores. Las buenas prácticas
de \textit{CI} \cite{8} permiten una rápida detección de errores y su resolución, un
\textit{feedback} rápido, la reducción de errores que provienen de tareas manuales, unas tasas de
\textit{commits} y \textit{pull requests} más altas, una calidad del \textit{software} mayor,
reconocer errores en producción temprano antes del despliegue, etc. Numerosos son sus ámbitos
de aplicación: \textit{software} empresarial, desarrollo de aplicaciones web, proyectos de código
abierto, aplicaciones móviles, etc. Todo ello, haciendo uso de las distintas herramientas que
existen en el mercado \cite{9}, como \textit{GitHub Actions}, \textit{Jenkins},
\textit{Travis CI}, \textit{CircleCI}, \textit{Azure DevOps}, entre otras.\\

El ciclo de vida de la Integración Continua, a pesar de ofrecer numerosas ventajas, conlleva
grandes costos asociados debido a los recursos computacionales \cite{10} necesarios para ejecutar
las construcciones, comúnmente denominadas \textit{builds}. A lo largo de este trabajo,
nos referiremos como costo computacional al hecho de ejecutar una \textit{build}, es decir, el
proceso de construir el \textit{software} y ejecutar todas las pruebas cuando la \textit{CI} es
lanzada. Este costo asociado se acentúa en empresas de gran tamaño, donde el número de
\textit{builds} que se ejecutan diariamente es muy elevado \cite{12,13}. Además, hay que sumar
la larga duración que pueden tene la ejecución de las \textit{builds} en este tipo de empresas.\\

En los últimos años, han surgido numerosos enfoques centrados en reducir el costo computacional
asociado a la ejecución de \textit{CI} \cite{1,2,4,5,6,7}. La idea principal de estos enfoques es
reducir el número de \textit{builds} que se ejecutan, prediciendo el resultado antes de su
ejecución y, por lo tanto ahorrándose ese costo computacional. Las \textit{builds} predichas como
construcciones exitosas (\textit{build pass}) no se ejecutan, mientras que las predichas como
construcciones fallidas (\textit{build failure}) sí se ejecutan. De esta forma, se mantiene el
valor conceptual de la \textit{CI}, que es la detección temprana de errores, pero reduciendo
el costo computacional asociado en el proceso. Este estudio toma como punto de partida el
algoritmo de \textit{machine learning} \textit{SmartBuildSkip} \cite{2}. La idea principal es
realizar una contribución a este algoritmo, usando features más significativas para predecir, o
bien realizando una variante propia del mismo que mejore los resultados obtenidos. Por lo tanto,
este estudio se enmarca en el desarrollo de software moderno, específicamente en el ámbito de
la Integración Continua y la predicción automática del resultado de dicha integración.\\

La memoria queda organizada de la siguiente forma: en primer lugar, se realiza un estudio del
estado del arte que sitúa los antecedentes previos a la Integración Continua y la predicción
automática de resultados de \textit{builds}. Posteriormente, se establecen los objetivos y
preguntas de investigación que pretende este estudio responder. A continuación, se describe
en detalle el problema a resolver, los principales obstáculos que se plantean y sus posibles
soluciones. Acto seguido, se desarrolla con detalle nuestra propuesta al problema, describiendo
las tecnologías usadas y el desarrollo de la solución. Después se presentarán las pruebas y
resultados obtenidos, comparando la solución con otras existentes, a modo de validar y verificar
la aportación de nuestra solución. Seguidamente, se comentarán las amenazas a la validez, una
parte esencial en cualquier trabajo de investigación. Este apartado nos permite identificar y
discutir posibles limitaciones que podrían afectar a la validez de los resultados y a las
conclusiones. Por último, se darán unas conclusiones sobre los resultados obtenidos y se
propondrán posibles líneas de trabajo futuro.
