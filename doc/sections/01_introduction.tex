\section{Introducción}
La Integración Continua (\textit{Continuous Integration, CI}) es una práctica de desarrollo de
\textit{software} que busca automatizar el proceso de fusión de cambios de código en un proyecto,
donde cada integración es verificada mediante la ejecución automática de pruebas. Este proceso
busca la detección temprana de errores y mejorar la calidad del software, permitiendo una
integración más frecuente y rápida del trabajo de todos los desarrolladores. Las buenas prácticas
de \textit{CI} \cite{8} permiten una rápida detección de errores y su resolución, un
\textit{feedback} rápido, la reducción de errores que provienen de tareas manuales, unas tasas de
\textit{commits} y \textit{pull requests} más altas, una calidad del \textit{software} mayor,
reconocer errores en producción temprano antes del despliegue, etc. Numerosos son sus ámbitos
de aplicación: \textit{software} empresarial, desarrollo de aplicaciones web, proyectos de código
abierto, aplicaciones móviles, etc. Todo ello, haciendo uso de las distintas herramientas que
existen en el mercado \cite{9}, como \textit{GitHub Actions}, \textit{Jenkins},
\textit{Travis CI}, \textit{CircleCI}, \textit{Azure DevOps}, entre otras.\\

Para contextualizar el problema que nos ocupa, vamos a describir algunos términos relevantes para
el entendimiento del mismo. A lo largo del trabajo, nos referiremos como \textit{build} al 
proceso automático mediante el cual el código fuente se compila, se ejecutan las pruebas, y se
genera un artefacto \textit{software}, ya sea un ejecutable, un contenedor, un paquete, etc., que
está listo para ser desplegado o usado en producción. Cada \textit{build} es lanzada por lo
que se denomina comúnmente \textit{trigger}, que puede ser un:
\begin{itemize}
    \item \underline{\textit{Commit}}: representa una ``instantánea'' del estado del proyecto en 
    un momento específico, guardando las modificaciones que se han hecho a los archivos desde el
    último \textit{commit}. Cada vez que el desarrollador realiza un \textit{commit}, se dispara
    una nueva \textit{build}.
    \item \underline{\textit{Pull request}}: un \textit{pull request} o solicitud de incorporación
    de cambios es una solicitud formal para fusionar cambios propuestos en una rama de desarrollo
    a otra rama, que generalmente es la rama principal. Este tipo de solicitud permite la
    revisión de los cambios realizados, su discusión, y aprobación del código por parte de
    otros desarrolladores antes de integrarlo con la rama principal. En este caso, al crear o
    actualizar un \textit{pull request}, se lanza una \textit{build} para verificar que el código
    cumple con los estándares de calidad.
    \item \underline{\textit{Schedule}}: se pueden programar \textit{builds} para que se ejecuten
    en un intervalo de tiempo regular, independendientemente de si hubo o no cambios en el código.
\end{itemize}

Existen numerosos sistemas de \textit{CI} en la actualidad, \textit{GitHub Actions},
\textit{Jenkins}, \textit{Travis CI}, \textit{CircleCI}, \textit{Azure DevOps}, entre otros, sin
embargo, en este trabajo nos centraremos en \textit{GitHub Actions}. \textit{GitHub Actions} es
el sistema de \textit{CI} más utilizado en la actualidad, y al cual muchos otros sistemas
migraron debido a sus características, especialmente \textit{Travis CI}. En 2020, \textit{Travis
CI} decidió imponer numerosas restricciones a su plan gratuito para proyectos \textit{software}
de código abierto \cite{9}, siendo este uno de los principales motivos para su migración hacia
\textit{GitHub Actions}. Además, existen otras razones para esta migración, como puede ser
utilizar una herramienta de \textit{CI} más confiable, mejor integración con soluciones 
\textit{self-hosted}, mejor soporte para múltiples plataformas, la reducción de la cantidad de
uso compartido de la herramienta, tener más funcionalidades, etc.\\

El ciclo de vida de la Integración Continua, a pesar de ofrecer numerosas ventajas, conlleva
grandes costos asociados debido a los recursos computacionales \cite{10} necesarios para ejecutar
las construcciones, comúnmente denominadas \textit{builds}. A lo largo de este trabajo,
nos referiremos como costo computacional al hecho de ejecutar una \textit{build}, es decir, el
proceso de construir el \textit{software} y ejecutar todas las pruebas cuando la \textit{CI} es
lanzada. Este costo asociado se acentúa en empresas de gran tamaño, donde el número de
\textit{builds} que se ejecutan diariamente es muy elevado \cite{12,13}. Ahorrar en dicho costo
computacional se convierte por tanto en un objetivo clave para las mismas. Optimizando la cantidad
de \textit{builds} que se ejecutan, podemos lograr una reducción significativa de este costo, ya
que se habrán consumido menor cantidad de recursos. Además, hay que sumarle el tiempo de espera
que los desarrolladores deben soportar cuando el tiempo de ejecución de la \textit{build} es
elevado, pudiendo ralentizar el tiempo de respuesta ante problemas y ajustes rápidos en el
desarrollo.\\

En los últimos años, han surgido numerosos enfoques centrados en reducir el costo computacional
asociado a la ejecución de \textit{CI} \cite{1,2,4,5,6,7}. La idea principal de estos enfoques es
reducir el número de \textit{builds} que se ejecutan, prediciendo el resultado antes de su
ejecución y, por lo tanto ahorrándose ese costo computacional. Las \textit{builds} predichas como
construcciones exitosas (\textit{build pass}) no se ejecutan, mientras que las predichas como
construcciones fallidas (\textit{build failure}) sí se ejecutan. De esta forma, se mantiene el
valor conceptual de la \textit{CI}, que es la detección temprana de errores, pero reduciendo
el costo computacional asociado en el proceso. Este estudio toma como punto de partida el
algoritmo de \textit{machine learning} \textit{SmartBuildSkip} \cite{2}. La idea principal es
realizar una contribución a este algoritmo, realizando un estudio de las \textit{features} que
se usan para la predicción, y añadiendo nuevas \textit{features} más significativas que puedan
mejorar estudios existentes. Además, se creará una aplicación web sencilla con la que el usuario
pueda interactuar de forma directa a través de una interfaz gráfica, abstrayendo la complejidad
de los algoritmos de predicción y ofreciendo una forma intuitiva y sencilla de realizar
predicciones basadas en un repositorio concreto. Por lo tanto, este estudio se enmarca en el
desarrollo de software moderno, específicamente en el ámbito de la Integración Continua y la
predicción automática del resultado de dicha integración.\\

La memoria queda organizada de la siguiente forma: en primer lugar, se realiza un estudio del
estado del arte que sitúa los antecedentes previos a la Integración Continua y la predicción
automática de resultados de \textit{builds}. Posteriormente, se establecen los objetivos y
preguntas de investigación que pretende este estudio responder. A continuación, se describe
en detalle el problema a resolver, los principales obstáculos que se plantean y sus posibles
soluciones. Acto seguido, se desarrolla con detalle nuestra enfoque al problema, describiendo
las tecnologías usadas y el desarrollo de la solución. Después se presentarán las pruebas y
resultados obtenidos, comparando la solución con otras existentes, a modo de validar y verificar
la aportación de nuestra solución. Seguidamente, se comentarán las amenazas a la validez, una
parte esencial en cualquier trabajo de investigación. Este apartado nos permite identificar y
discutir posibles limitaciones que podrían afectar a la validez de los resultados y a las
conclusiones. Por último, se darán unas conclusiones sobre los resultados obtenidos y se
propondrán posibles líneas de trabajo futuro.
