\section{Amenazas a la validez}
En cualquier trabajo de investigación, es crucial reconocer los factores que pueden comprometer
la validez de los resultados obtenidos. Estos factores, comúnmente conocidos como amenazas a la
validez, representan posibles fuentes de sesgo o error que pueden afectar a la precisión y
generalización de las conclusiones del estudio. Al identificar y discutir estas amenazas, no solo
hacemos más transparente nuestro estudio, sino que también proporcionamos una base crítica para
que los lectores evalúen la fiabilidad de los hallazgos. En este apartado se examinan las
principales amenazas que podrían afectar a la validez de constructo, validez interna y validez
externa del presente estudio, con el fin de contextualizar los resultados y ofrecer una
interpretación más solida y matizada.\\

Concretamente, se consideran tres tipos de amenazas a la validez: validez de constructo, validez
interna y validez externa. A continuación, explicamos en qué consiste cada una de ellas:

\begin{itemize}
    \item \textbf{Validez de Constructo}: son aquellos factores que ponen en duda si el estudio
    está realmente midiendo lo que pretende medir, es decir, si los conceptos definidos en la
    investigación son evaluados correctamente. Estas amenazas pueden afectar la interpretación
    de los resultados en relación con los constructos teóricos.\\

    \item \textbf{Validez Interna}: se refiere a las amenazas que pueden afectar la relación
    causal entre las variables independientes y dependientes. En otras palabras, se trata de
    factores que pueden distorsionar la interpretación de la relación entre las variables
    manipuladas y las variables de respuesta.\\

    \item \textbf{Validez Externa}: son aquellas amenazas que pueden afectar la generalización de
    los resultados obtenidos en el estudio. Estas amenazas se refieren a la capacidad de
    generalizar los resultados a otros contextos, poblaciones o situaciones distintas a las
    evaluadas en el estudio.
\end{itemize}

A continuación, se detallan las amenazas a la validez identificadas en el presente estudio y se
discuten las estrategias utilizadas para mitigar su impacto en los resultados obtenidos.

\subsection{Validez de Constructo}
En nuestro estudio, utilizamos métricas como indicadores para representar la cantidad de
\textit{build failures} detectados en CI. Algunas métricas, como el \textit{accuracy}, puede ser
engañosa en conjuntos de datos que están muy desbalanceados, ya que puede reflejar un alto
valor incluso cuando el modelo falla en identificar los \textit{build failures}. Para mitigar
este problema, hemos utilizado métricas como \textit{precision}, \textit{recall} y \textit{F1-score},
que proporcionan una visión más equilibrada del rendimiento del modelo. Además, debemos
recordar que nuestro enfoque persigue reducir el costo computacional asociado a la CI mediante
la detección de \textit{build failures}, sin embargo, no hemos considerado valores 
económicos específicos asociadas a la ejecución de CI o a la reparación de errores en el conjunto
específico que hemos usado para la experimentación.


\subsection{Validez Interna}
Para salvaguardar la validez interna, hemos realizado pruebas exhaustivas de nuestros
procedimientos de evaluación en subconjutos del \textit{dataset} empleado durante el desarrollo.
Nuestro análisis puede estar influenciado por información incorrecta en nuestro dataset. Por
esto, siempre hemos considerado ramas principales de los proyectos, filtrando así
valores atípicos más comunes en ramas secundarias o de desarrollo. Además, nuestros resultados
podrían verse afectados por pruebas inestables (\textit{flaky tests}) que causan fallos
de manera errática o falsa.\\

Por último, la validación cruzada cronológica ha sido empleada para preservar el orden temporal
de los datos y evitar que futuras \textit{builds} se incluyan en el conjunto de entrenamiento. Esto
garantiza que los resultados sean más representativos de cómo el modelo funcionaría en
condiciones reales, donde el entrenamiento se realiza con datos pasados y la prueba se realiza
con datos futuros. Aunque la validación cruzada cronológica es una alternativa adecuada a la
validación cruzada estándar, la decisión de utilizar este enfoque ha sido alineada con el objetivo
de evaluar las técnicas en un escenario que respete la secuencia temporal.


\subsection{Validez Externa}
Para aumentar la validez externa, hemos seleccionado $20$ proyectos ampliamente conocidos de
\textit{GitHub}. Todos son proyectos de código abierto y que tienen un número considerable de
\textit{forks} y estrellas. Como dato objetivo, el proyecto que menos \textit{forks} y estrellas
tiene es \textit{coherence}, de \textit{Oracle}, con $70$ \textit{forks} y $427$ estrellas, y el
que más, \textit{spring-boot}, con $40,500$ \textit{forks} y $74,400$ estrellas. Los proyectos
seleccionados son todos proyectos \textit{Java}, ya que necesitamos hacer una evaluación justa
con otras técnicas. A pesar de que este lenguaje de programación es uno de los más utilizados 
en la actualidad, lenguajes de programación diferentes pueden tener hábitos de Integración
Continua distintos, pudiendo provocar resultados ligeramente diferentes a los obtenidos en este
estudio. \\

Finalmente, haber elegido proyectos de empresas muy conocidas puede haber limitado la
diversidad del conjunto de datos, ya que estos proyectos suelen seguir procesos de desarrollo de
\textit{software} altamente estructurados. Esto podría no reflejar la realidad de proyectos más
pequeños o menos formales, pudiendo limitar la aplicabilidad de los resultados a repositorios
menos formales o con menos recursos.

