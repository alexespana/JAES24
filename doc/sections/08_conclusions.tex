\section{Conclusiones y trabajos futuros}
En este trabajo, hemos propuesto y evaluado JAES24, un enfoque novedoso para ahorrar
costos en la Integración Continua (CI) al omitir \textit{builds} que se predice que pasarán.
Nuestro diseño de JAES24 parte de la hipóstesis de que el momento en que se realiza una
contribución a un proyecto de \textit{software} influye en la probabilidad de que la \textit{build}
pase la CI o no. Además, parte también de que el desgranado de los tipos de cambios que se
realizan en una \textit{build} son especialmente importantes para la predicción de su resultado.
Estudiamos la relación entre el momento de la contribución y el resultado de la CI y encontramos
evidencia que las apoya. Se ha descubierto que: el día, la hora o el tiempo entre contribuciones
influye en la probabilidad de predecir un \textit{build failure}. Además, otros cambios como el
número de líneas añadidas o eliminadas, que desgranan el tipo de cambio producido, también tienen
un impacto positivo en la predicción de \textit{build failures}.\\

Con este conjunto de características, JAES24 mejoró el \textit{precision} y el
\textit{recall} de \textit{SmartBuildSkip}. Además, se trata de un algoritmo personalizable, ya
que puede configurarse con distintos umbrales de decisión en función de las necesidades de los
desarrolladores y, además, cuenta con dos versiones: una más conservadora y otra más agresiva.
La variante conservadora, JAES24-\textit{Within}, por lo general, detecta un menor número de
\textit{build failures} pero mantiene unos niveles de \textit{precision} más elevados. Por otro
lado, la variante más agresiva, JAES24-\textit{Without}, detecta un mayor número de \textit{build
failures}, sin embargo, mantiene unos niveles de \textit{precision} menores, lo cuál significa que
a menudo predice como \textit{build failures} \textit{builds} que en realidad pasan (falsos positivos). Desde
el punto de vista del desarrollador, que JAES24 incluya además, una interfaz gráfica, hace
que sea mucho más fácil de usar y entender, convirtiéndose en un enfoque único y más accesible
para los desarrolladores. JAES24 proporciona una estrategia novedosa que complementa las
técnicas existentes para ahorrar costos en CI, omitiendo construcciones con cambios que no afectan
al código.\\

En el futuro, trabajaremos extendiendo la funcionalidad de JAES24 para que pueda
realizar análisis estáticos en función del contenido de los cambios realizados en el código
fuente de las \textit{builds}. Igualmente, otros tipos de análisis como el dinámico
pueden ser considerados. Además, lo haremos extensible a otros lenguajes de programación, estudiando los hábitos
de CI que puedan darse en proyectos \textit{software} de distinta naturaleza. Podría
realizarse un código totalmente funcional, que permitiera a los desarrolladores integrar
JAES24 en su ciclo de desarrollo \textit{software}, permitiendo que puedan hacer
predicciones basándose en su repositorio local. Finalmente, se realizarán cambios en la
interfaz gráfica para hacerla más flexible y accesible a los desarrolladores, incluyendo nuevas
funcionalidades y mejorando la usabilidad.\\