\section{Descripción del problema}
La Integración Continua (CI) es una práctica esencial en el desarrollo de \textit{software}
moderno, que busca automatizar la fusión de cambios de código mediante la ejecución de pruebas
automáticas. Esta práctica permite detectar errores de forma temprana y mejorar la calidad del
\textit{software}, facilitando una integración más frecuente y rápida del trabajo de los
desarrolladores. Aunque la CI ofrece numerosas ventajas, su implementación conlleva
significativos costos computacionales asociados, especialmente en empresas de gran tamaño donde
se ejecutan un elevado número de \textit{builds} diariamente. Estos costos no solo incluyen el
costo de recursos de cómputo para ejecutar las \textit{builds}, sino tmabién el tiempo de espera
al que se pueden enfrentar los desarrolladores durante el proceso de integración. \\

El objetivo principal de este trabajo es abordar el problema de la optimización de costos
computacionales en CI mediante la predicción automática del resultado de las
\textit{builds}. Se utilizarán algoritmos de \textit{Machine Learning} para predecir si una
\textit{build} concreta pasará o fallará antes de ser ejecutada. En especial, nos centraremos
en predecir las \textit{builds} que fallan, ya que son las más valiosas para los desarrolladores
y de las cuales depende la puesta en producción o no del \textit{software}. Este enfoque
permite mantener la detección temprana de errores, que es el objetivo fundamental de CI,
mientras se reduce el consumo de recursos computacionales.\\

Por lo tanto, este estudio se centra en desarrollar un modelo predictivo que permita optimizar
el proceso de Integración Continua. Se pretende:

\begin{itemize}
    \item Implementar un algoritmo de aprendizaje automático que permita predecir los resultados
    de las \textit{builds} basándose en un conjunto de características. Se probarán varios
    algoritmos para determinar cuál es el que mejor resultados ofrece en nuestro problema.\\
    \item Analizar y seleccionar las características más significativas que influyen en la
    predicción. Para lo cual, se realizará un estudio de las \textit{features} presentadas en
    otros estudios y se añadirán otras que puedan ser relevantes para nuestro problema.\\
    \item Evaluar la importancia que cada \textit{feature} tiene sobre el modelo de predicción, lo
    que permitirá a los desarrolladores identificar qué aspectos de las \textit{builds} son
    más significativos en la predicción de su resultado.
\end{itemize}
