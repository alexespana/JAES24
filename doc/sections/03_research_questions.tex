\section{Objetivos y preguntas de investigación}
En un estudio de carácter exploratorio como el que se propone, definir unos objetivos y preguntas
de investigación se convierte en una tarea fundamental para la correcta orientación del trabajo.
En este sentido, los objetivos nos permiten establecer una serie de metas a alcanzar, mientras
que las preguntas de investigación nos ayudan a centrar el estudio en aspectos concretos que
queremos responder. Los objetivos de la investigación son los siguientes:

\begin{itemize}
    \item \textbf{OB-1}: implementar un algoritmo de aprendizaje automático que genere un modelo
          predictivo (un \textit{predictor}) basado en un conjunto de características
          \textit{features} extraídas de las \textit{builds}.\\
    \item \textbf{OB-2}: utilizar la \textit{API} de GitHub para obtener datos relevantes sobre
          las \textit{builds}, como su histórico, características asociadas, resultados anteriores
          de la integración continua.\\
    \item \textbf{OB-3}: desarrollar e implementar diferentes algoritmos de predicción con la
          selección de diferentes características con el objetivo de proporcionar múltiples
          opciones a la hora de predecir el resultado de la integración continua.\\
    \item \textbf{OB-4}: implementar una interfaz gráfica que sirva como punto de entrada de
          datos para el algoritmo de predicción y que permita visualizar los resultados
          obtenidos.
\end{itemize}

Las preguntas de investigación tienen el objetivo de estructurr y orientar el proceso de
investigación. A continuación se establecen las preguntas de investigación junto a las
métricas usadas para su evaluación:


\begin{itemize}
    \item \textbf{PI-1}: ¿Qué algoritmo de predicción produce los mejores resultados en la
          predicción automática del resultado de la integración continua?\\
          \begin{itemize}
            \item \textbf{Métrica}: \textit{accuracy}, \textit{precision}, \textit{recall} y
                  \textit{F1-score} del modelo.\\
          \end{itemize}

    \item \textbf{PI-2}: ¿Qué características de las \textit{builds} son más significativas en
          la predicción?\\

          \begin{itemize}
            \item \textbf{Métrica}: importancia de cada feature a través de la interpretación de los
                  coeficientes del modelo.
          \end{itemize}

\end{itemize}
